\documentclass[a4paper,10pt,twocolumn,uplatex]{jsarticle}
\usepackage{style/nislab}

%---------------------------------------------------------------------
% レジュメ種別・日付設定(要変更)
% \type{} 1:修士論文諮問会 2:卒業論文発表会 else:月例発表会
\type{3}
\year{2021}
\month{4}
\date{1}

%---------------------------------------------------------------------
% ページ番号設定(要変更)
\setcounter{page}{1}

%---------------------------------------------------------------------
\begin{document}
%---------------------------------------------------------------------
% タイトル作成部分(要変更)
% \maketitle{タイトル}{title}{名前}{name}
\maketitle{タイトルタイトルタイトルタイトルタイトルタイトル}
{Title Title Title Title Title Title}
{竹内 一真}
{Kazuma Takeuchi}

%---------------------------------------------------------------------
\section{はじめに}
近年,多方面でのドローンを活用した事業が進出しており,屋内での小型ドローンの利用も期待されている.しかし,狭小空間でのドローンの飛行は,障害物が多く,操縦者から見えない場所であったりと,遮られた視点からの操縦を必要とし,操縦は困難が懸念される.
\par
そこで,拡張現実を用いることで,操縦者の死角領域内を可視化し,狭小空間での操縦性の向上を図る手法を提案した.
また,操縦者視点の操縦を実現する上で,障害物までの距離感が掴めないことが懸念されている.
そこで,ドローン近傍の障害物を検知するデザイン案を提案することで,操縦者にとってどのような情報が障害物までの距離感を掴むのに適しているかを検討した結果、ARを用いた手法ではARなしの手法と比較して、操縦時間と衝突回数が低いことがわかった.
しかし,実験に費やした時間が平均的に短いことから実際の場面で使用することを考えると,操縦者目線のみでのドローン操縦ではドローン周辺の環境を完全に認識できるわけではないので,ドローン操縦における安全性の不足が考えられる.
本研究では,ARにより表示されたドローン及びその周辺の環境を、複数人でリアルタイムに確認できる手法を提案する.これにより、操縦者一人のみの場合と異なり、よりドローンの安全性を向上させることを目指す.

%---------------------------------------------------------------------
% \section{図表のベストプラクティス}
% \LaTeX{}を使いこなすにあたり,図表の活用は重要である.基本的にはLaTex Wiki\cite{latex_wiki}を参考にすれば問題ない.\par
% \section{関連研究}
% \subsection{}
% Eratらの研究では\cite{drone},狭い場所でのドローン操縦が困難ということで,三人称視点のドローン操縦手法を提案している.ドローンがSLAM技術で空間マッピングをすることで,空間の仮想現実を構築し,閉鎖環境をみえるようにするというものである.しかし,空間の環境は事前に準備されたものであり,動的に再構築することがないので利便さに欠ける.
% \subsection{概要}
% またMichaelらは\cite{interface},人間がドローンの動きの意図を視覚的に理解するため,ARを用いたユーザインタフェースデザインを作成し評価した.結果として,ドローンに対しARを用いた様々なデザインは,ARなしと比べ,課されたタスク効率を大幅に向上させ,ARを用いることでドローンの操縦性を向上させるための直感的で視覚的な合図を提供することが可能であることが示された.

%---------------------------------------------------------------------
% \subsection{図}
% 図を挿入する場合は,図\ref{fig:sample1}や\figref{fig:sample2}のように引用することができる.図の横幅が大きい場合は,\figref{fig:sample2}のようにすることもできる.\par
% ちなみに,\LaTeX{}ではベクターファイルとしてEPSファイルを推奨していた頃もあったようだが,現在はPDFファイルを使用することが推奨されている.PDFファイルに出力するのが前提なら,dvipdfmxではPDF,PNG,JPEG がそのまま使用できる.dvipdfmxはEPSファイルそのものを自分で扱えないので,Ghostscriptを内部で呼び出して変換する.PDFファイルで問題がなければEPSにこだわる必要はないと思われる.ただし,ジャーナルによっては図としてPDFを使うのがダメだったりするので慎重に.


\section{関連研究}
\subsection{}
Anhongらの研究では\cite{collaborative},未だARは,エンドユーザーが消費するコンテンツを作ることができないため,永続的なAR構造を共同で作成することができるモバイルアプリケーションを提案し,実験協力者が同一空間,異なる空間,異なる空間の上,異なる時間の3つの環境の上で,どの環境における共同作業が最も好まれるかを評価した.
結果として,同一空間で共同作業を行う環境が最も好まれたため,本研究においても同一空間での共同作業を行うものとする.

%---------------------------------------------------------------------

\begin{figure}[!tb]
  \centering
  \includegraphics[width=\linewidth]{img/sample1.pdf}
  \caption{悩む男の子}
  \label{fig:sample1}
\end{figure}

\begin{figure*}[!tb]
  \centering
  \includegraphics[width=\linewidth]{img/sample2.pdf}
  \caption{ドライブする家族}
  \label{fig:sample2}
\end{figure*}

%---------------------------------------------------------------------

\section{提案手法}\label{discussion}
\subsection{死角領域内のAR可視化}
本研究では,操縦者とドローンの間に障害物があり,ドローンを視認できない環境を想定する.障害物が存在すると判断した際,その障害物を透過することで,操縦者への死角領域の空間認識を提供する.
また,死角領域内をドローンが飛行している際に,近傍の障害物までの距離が掴めない問題点を解決するために,2つのARインタフェースデザインを提案した.

\subsection{Stereo}
Stereoのデザインは,ステレオビジョンを参考にして,ドローンから障害物までの距離に応じて,障害物の色を分けている.Stereoは,全体的な環境の理解を提供しており,ドローン周辺の障害物全ての衝突の危険性を示す.

\subsection{Marker}
Markerのデザインは,ドローンから見て最も近い障害物に対して,目印を付けている.Stereoでは障害物全てが色分けされているため,操縦者を混乱させる可能性がある.Markerでは,最も危険な障害物だけを知覚させるため,Stereoに比べ簡易的なアプローチとなっている.

\subsection{複数人でのAR共有}
本研究では,死角領域内のAR可視化を行った上で,可視化した環境地図,ドローンを複数人でリアルタイムに視認できる仕組みを構築する.システム構成を に示す.
図 のように各端末が単一のARマーカーを参照することで,ARマーカーを三次元のワールド座標(X,Y,Z)と想定し,マーカーとの相対位置関係により,それぞれの端末の位置情報を導き出す.
この際,端末で映し出したARマーカーまでの距離を,HoloLens搭載の1-MP ToF (Time of Flight) 深度センサーにより取得する.
取得した各端末の位置情報,角度をクラウドに送信し,3次元環境地図内における各ユーザの位置合わせを行う.

\subsection{動作手順}
\begin{enumerate}
  \item 書かれた論文は書いた人の研究者としての人格を表す
  \item データのみ出して論文を書かない者は,テクニシャンである
  \item データも出さず,論文(原著論文)を書かない者は,評論家である
  \item 研究者は論文を書くことによって成長する.また,成長の糧にしなければならない
  \item 論文は研究者の飯のタネである
\end{enumerate}


%---------------------------------------------------------------------

\section{評価}\label{experiment}
\subsection{実装}

%---------------------------------------------------------------------
\section{まとめと今後の課題}
小型ドローンでの遮られた視点からの狭小空間での操縦は死角の多さや,ドローンと障害物までの距離感が測れないことが懸念され,本研究では操縦者の死角領域内に存在するドローンと周辺を可視化し,ドローン周辺の障害物を知覚するためのARデザインを提案し,実験を行うことで遮られた視点からの狭小空間でのドローン操縦性を評価した.結果として,ARを利用した手法では実験環境での操縦時間が短く,衝突回数も少なかったことから操縦性の向上が確認された.
また,障害物を知覚するためのARデザインでは,ドローン周辺の障害物に危険度を振り分けている手法が,操縦者への操縦への安心を与え,操縦性を向上させたことが確認できた.

%---------------------------------------------------------------------


% \subsection{表}
% 表は\tabref{tab:data_type}のように引用することができ,表を作成する場合は罫線を少なくすることと,横線のみの使用を心がけることが推奨される.

% \begin{table}[!bt]
%   \caption{代表的なデータの型}
%   \label{tab:data_type}
%   \centering
%   \begin{tabular}{lcr}
%     \hline
%     データの型         & 宣言   & ビット幅 \\
%     \hline \hline
%     短整数型           & short  & 16       \\
%     整数型             & int    & 32       \\
%     単精度浮動小数点型 & float  & 32       \\
%     倍精度浮動小数店型 & double & 64       \\
%     \hline
%   \end{tabular}
% \end{table}

%---------------------------------------------------------------------
% \section{研究者にとっての論文十箇条}
% 論文を書くことは大切だ必要だ,と周囲から言われる.それは自分でも分かっているつもりだけれど,その理由をはっきりと伝えてもらえる機会は少ない.研究者にとっての論文十箇条\cite{whats_paper}は,とてもシンプルでわかりやすく,非常に心にきた.一度目を通してみるべきであろう.

% \begin{enumerate} % 箇条書きは \begin{itemize}
% \item 書かれた論文は書いた人の研究者としての人格を表す
% \item データのみ出して論文を書かない者は,テクニシャンである
% \item データも出さず,論文(原著論文)を書かない者は,評論家である
% \item 研究者は論文を書くことによって成長する.また,成長の糧にしなければならない
% \item 論文は研究者の飯のタネである
% \item 論文は後世の研究に影響を与えなければならない
% \item 研究者は書いた論文に責任を問われる
% \item 忙しくて論文が書けないというのは,言い訳にはならず,能力がないといっているのと同じである
% \item 博士論文以上の論文を書けない者は,その博士論文は指導教官のものといわれても仕方がない
% \item 研究において最も重要なのはアイデアであり,それが試されるのが論文である
% \end{enumerate}

%---------------------------------------------------------------------
% Bibliography
\footnotesize{
  \begin{thebibliography}{99}
    \bibitem{latex_wiki} Latex Wiki (\url{https://texwiki.texjp.org/}).
    \bibitem{whats_paper} 渡辺 豊, "角皆静男先生のご逝去を悼む", 地球化学, vol.50, no.1, pp.1-3, 2016.
  \end{thebibliography}
}

%---------------------------------------------------------------------
\end{document}
%---------------------------------------------------------------------
