%%%%%%%%%%%%%%%%%%%%%%%%%%%%%%%%%%%%%%%%%%%%%%%%%%%%%%%%%%%%%%%%
% sample.tex
% 情報処理学会全国大会論文 サンプルファイル
%
%  Susumu ISHIHARA Jan 2000
%  ishihara@cs.inf.shizuoka.ac.jp
%%%%%%%%%%%%%%%%%%%%%%%%%%%%%%%%%%%%%%%%%%%%%%%%%%%%%%%%%%%%%%%%
\documentclass[a4j,10pt,twocolumn]{jarticle}
\usepackage{ipsjz}           % 全国大会用パッケージ
\usepackage[dvips]{graphics} % 図の取り込み用
%%%%%%%%%%%%%%%%%%%%%%%%%%%%%%%%%%%%%%%%%%%%%%%%%%%%%%%%%%%%%%%%
%% タイトル
\title{情報処理学会全国大会論文サンプル} 
\etitle{A Sample of a IPSJ proceeding}   

%% 著者名
\author{静大太郎\DAG \quad 静大二郎\DAG \quad 名大花子\DDAG}
\eauthor{Taro SHIZUDAI\DAG, Jiro SHIZUDAI\DAG and Hanako MEIDAI\DDAG} 

%% 所属
\affiliation{\DAG 静岡大学情報学部 \quad\quad \DDAG 名古屋大学工学部}
\eaffiliation{\DAG Faculty of Information, Shizuoka University\\
432-8011, Hamamatsu, Japan\\
\DDAG Dept. of Engineering, Nagoya University\\
464-8603, Nagoya, Japan}

%% メールアドレス
\email{\{address1, address2, address3\}@cs.inf.shizuoka.ac.jp}

%%%%%%%%%%%%%%%%%%%%%%%%%%%%%%%%%%%%%%%%%%%%%%%%%%%%%%%%%%%%%%%%
\begin{document}
\maketitle              % 日本語タイトル作成
\makeetitle             % 英語タイトル作成
%\baselineskip=1.53zh   %必要に応じて行間を調整

\section{はじめに}

このドキュメントは,アスキー版日本語 \LaTeXe\cite{texdoc}\cite{ascii}と
情報処理学会全国大会論文のためのipsjz パッケージを使って論文を作成する
ためのサンプルです。ipsj.sty は 1999 年秋の全国大会の書式に基づいて作成
されていますが,今後,情報処理学会の規定する書式の変更によって,このパッ
ケージは予告なく変更されます。

\section{所属等の書き方}

サンプルを見れば分かると思いますが,ipsjz パッケージには jarticle クラ
スで提供されている \verb+\title+ や \verb+\author+ に加えて,
\verb+\etitle+,\verb+\eauthor+,\verb+\affiliation+,
\verb+\eaffiliation+,\verb+\email+ が追加されています。それぞれ,英文
タイトル,英文著者名,和文所属,英文所属,メールアドレスを指定するコマ
ンドです。これらの情報は \verb+\maketitle+ および \verb+\makeetitle+ コ
マンドで出力されます。なお,\verb+\makeetitle+ コマンドは 
\verb+\maketitle+ コマンドより後に使用してください。

\section{図の取り込み}

\subsection{図のデータ形式}

図のデータは EPS で作成するのが良いでしょう。Tgif や Illustrator で作成
するのが良いでしょう。Windows の印刷で EPS 出力して作成した EPS ファイ
ルは,サイズ調整がうまくいかないことが多いと感じています。

\subsection{図の取り込みの実例}

図の取り込みには graphics パッケージによって提供される 
\verb+\includegraphics+ コマンドを使用します。\textbf{図~\ref{fig-sample1}}
では,図をそのままの大きさで取り込んでいます。

\begin{figure}[tb]
 \begin{center}
  \includegraphics{fig1.eps}
  \caption{図のサンプル(サイズ調整なし)}
  \label{fig-sample1}
\end{center}
\end{figure}

\textbf{図~\ref{fig-sample2}} では,図の横幅を \verb+\columnwidth+ に合
わせています。サイズの調整には graphics パッケージで提供されている 
\verb+\resizebox+ コマンドを使用しています。

figure 環境での図の配置指定には [h] を使わないようにしてください。図に
よって文章が分断され,文章が読みにくくなることがあるからです。位置指定
には [tb] [t] [b] のいずれかを指定してください。

\begin{figure}[tb]
 \begin{center}
  \resizebox{\columnwidth}{!}{\includegraphics{fig1.eps}}
  \caption{図のサンプル(サイズ調整版)}
  \label{fig-sample2}
\end{center}
\end{figure}


\section{美しい組版をするために}

美しい組版をするために以下のことに気をつけましょう。

\begin{itemize}
 \item \textbf{括弧を適切に使用する} 日本語文中では全角の()を,英文中では半角
       の () を使用するときれいに仕上がります。全角と半角の括弧では,括
       弧の下の位置が異なります。
 \item \textbf{英文と和文の間には半角のスペースをいれる}
 \item \textbf{数式は数式モードで書く} \TeX の機能を十分に使わないばか
       りか,フォントの違いによって数式の意味が分かりにくくなってしまい
       ます。数式モード中でのフォントの使い方にも気をつけてください。
 \item \textbf{全角スペースを使った位置調整は行わない} 折角の \TeX の高
       度な組版機能が発揮できません。スペース調整は \TeX の機能をうまく
       使って行うようにしましょう。参考文献に挙げた \cite{texdoc} など,
       一冊は手元に \LaTeXe 関係の本を用意しておくと良いでしょう。(個
       人的には \TeX ブック \cite{texbook} がお勧めだけど,玄人向けだな
       あ)
\end{itemize}

\section{おまけの機能}

ipsjz パッケージには論文を書きやすくするためのおまけ機能があります。
\textbf{表~\ref{tbl-addon}} を参照してください。これらの機能は,作者の
気分によって予告なく追加されます。

\begin{table}[tb]
 \begin{center}
  \caption{おまけコマンド}
  \label{tbl-addon}
  \small
  \begin{tabular}{lp{5cm}}
   \Hline
   \multicolumn{1}{c}{\textbf{コマンド}} & 
   \multicolumn{1}{c}{\textbf{機能}} \\
   \hline
   \verb+\Hline+ & この表の一番上の罫線のように,太い罫線を描く。\\
   \verb+\DAG+ & 上付のダガー(\DAG)。所属等の区分に使用。\\
   \verb+\DDAG+ & 上付のダブルダガー(\DDAG)。所属等の区分に使用。\\
   \verb+\slashbr{}+ & スラッシュ(/)でも改行できるようにする。URL を参
   考文献に使用するときに重宝する。このドキュメントでも参考文献 
   \cite{ishihara} のところで使用している。\\
   \Hline
  \end{tabular}
 \end{center}
\end{table}

\section{まとめ}

以上,ipsjz パッケージによる論文の書き方を簡単に紹介しました。このパッ
ケージを使ってばりばり論文を書いてくださることを期待します。なお,私の
ホームページでも簡単な \LaTeX の使い方の説明のページ \cite{ishihara} が
あるので参考にしてください。

\begin{thebibliography}{9}
%\baselineskip=1.3zh       %必要に応じて行間を調整
 \bibitem{texdoc} 中野: 日本語 \LaTeXe ブック, アスキー出版局, 1996
 \bibitem{ascii} {http://www.ascii.co.jp/pb/ptex/index.html}
 \bibitem{texbook} D.~E.~Knuth: 改定新版 \TeX ブック, アスキー出版局,
         1992
 \bibitem{ishihara}\slashbr{
         http://apus.cs.inf.shizuoka.ac.jp/\~ishihara/LaTeX/index.html}
\end{thebibliography}
\end{document}
